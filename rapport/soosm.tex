\documentclass[a4paper, 11pt]{article}

\usepackage[utf8]{inputenc}
\usepackage[T1]{fontenc}
\usepackage[french]{babel}
\usepackage{helvet}
\renewcommand{\familydefault}{\sfdefault}
\usepackage{blindtext}
\usepackage{fullpage}
\usepackage{xspace}
\usepackage{verbatim}
\usepackage{graphicx}
\usepackage{listings}
\usepackage[usenames,dvipsnames]{xcolor}
\usepackage{url}
\usepackage{float}
\usepackage{caption}

\title{Simulation Orientée-Objet de systèmes multiagents}
\author{Equipe 3}
\date{Novembre 2017}

\begin{document}

\maketitle

\section{Simulateur de balles}

\subsection*{Choix de conception}

Le simulateur de balles est représenté par la classe \textit{BallsSimulator} et réalise l'interface \textit{Simulable}. Cette classe conserve une liste des balles \textit{Ball} initialisées aléatoirement et à simuler. On utilise le gestionnaire d'événements par la classe interne \textit{UpdateBalls}.

\subsection*{Tests effectués}

\textit{TestBalls} : affiche une liste de \textit{Ball} dans la console.

\textit{TestBallGUI} : simulation de balles. Lorsqu'une balle touche un bord, elle rebondit.

\section{Automates cellulaires}

\subsection*{Choix de conception}

La classe abstraite \textit{CellularAutomaton} représente un système de type automate cellulaire. Les automates cellulaires Jeu de la Vie de Conway et Modèle de Schelling en sont des classes dérivées, respectivement \textit{GameOfLife} et \textit{SchellingModel}. Le Jeu de l'Immigration est représenté par la classe \textit{ImmigrationGame} et est dérivé du Jeu de la Vie. Les trois classes d'automates cellulaires réalisent l'interface \textit{Simulable}. Tous ces automates sont initialisés de manière aléatoire.

\subsubsection*{\textit{States}}

Cette classe permet la gestion de la grille de cellules et l'association des états aux \textit{Color}.

La grille est conservée sous forme d'une matrice d'états \textit{State} dans la classe \textit{States}. La classe \textit{State} conserve l'état initial d'une cellule ainsi que son ancien et son nouvel état relativement au temps \textit{t} de la simulation. Cela permet d'éviter une éventuelle perte d'information si l'on modifiait l'état des cellules de la grille "à la volée". On calcule le nombre de voisins grâce à ses états.

La classe \textit{States} conserve également un tableau de \textit{Color} donné en paramètre par l'utilisateur à l'initialisation. Un état correspond à un indice du tableau et on peut donc associer chaque état à une \textit{Color}.

\subsubsection*{\textit{CellularAutomaton}}

Cette classe regroupe les fonctions communes à tous les automates cellulaires : initialisation et ré-initialisation de la grille de cellules. Cette classe conserve également le positionnement de la grille en pixels et permet de faire le lien entre la matrice d'états conservée dans \textit{States} et la grille affichée.

\subsubsection*{\textit{GameOfLife} \& \textit{ImmigrationGame}}

Pour ces deux automates, on réinitialise l'affichage graphique à chaque étape de la simulation. Comme on ne peut pas changer la couleur d'un \textit{Rectangle}, on les recrée tous au fur et à mesure que l'on met à jour leur état. Même si après quelques itérations, il y peu de cellules qui continue à changer d'état dans le Jeu de la Vie, nous avons décidé de réinitialiser l'affichage plutôt que de rajouter des \textit{Rectangle} par dessus les cellules changeant d'état. Cela permet d'éviter qu'il y est une accumulation du nombre d'objets à afficher et donc de réduire l'impact mémoire. De plus il n'y a pas de latence à l'affichage avec cette méthode.

\subsubsection*{\textit{SchellingModel}}

Les habitations sont représentées par des \textit{Cell}. Une \textit{Cell} est composée de ses coordonnées dans la matrice d'états de \textit{States} et de son \textit{Rectangle}. La couleur des habitants est représentée par son état dans la grille de \textit{States}. Un état égal à zéro représente une habitation vacante. Les habitations occupées sont conservées dans une liste et les habitations vacantes dans une file de priorité.

A chaque étape de la simulation, on parcours la liste, calcule le nombre de voisins et éventuellement déménage la famille. Le déménagement s'effectue par une translation des \textit{Cell} représentant l'habitation occupée et l'habitation vacante récupérée dans la file, chacune à la place de l'autre. L'habitation vacante qui se situe maintenant à la place de l'ancienne habitation occupée est replacée dans la file.


\subsection*{Tests effectués}

\textit{TestGameOfLife} : simulation d'un Jeu de la Vie de Conway. Les cellules vivantes sont en bleu et les cellules mortes en blanc.

\textit{TestImmigrationGame} : simulation d'un Jeu de l'Immigration. Le cycle de couleur est : blanc $\rightarrow$ jaune $\rightarrow$ orange $\rightarrow$ rouge $\rightarrow$ blanc $\rightarrow$ etc.

\textit{TestSchellingModel} : simulation du Modèle de Schelling. On utilise dans cette simulation 5 couleurs (rouge, bleu, vert, jaune, rose) et un seuil $k$ = 5. On remarque que même au seuil $k$ = 7, de petites communautés de même couleur se forme. Plus $k$ est petit, plus ces communautés sont grosses et tendent à être unique pour chaque couleur.

\section{Modèle des boids}

\subsection*{Choix de conception}
Nous avons implémenté une classe Boid simple contenant les caractéristiques d'un boid: sa masse, sa vitesse maximale, ses vecteurs de position, de vitesse ainsi que d'accélération.

Pour nous aider dans les opéréations géométriques, nous avons aussi implémenté une classe Vect, qui contient tout ce qui est utile pour manipuler ces trois vecteurs.

La classe Boids contient un ensemble de boids, s'occupe d'animer les boids en utilisant le principe fondamental de la dynamique.
Nous avons ajouté par rapport aux trois forces de départ une force permettant de garder les boids dans l'écran (ou du moins les faire revenir) et une autre permettant de les faire tourner en cercle si ils ne trouvent pas de voisins visibles de la même espèce.

Il suffit d'appeler des setters adéquats pour configurer les boids afin de leur donner aisément un comportement particulier, qui peut fortement varier.

Une autre classe, Predators hérite de Boids pour représenter les prédateurs. Ceux-ci cherchent non pas des voisins au sein de leur propre groupe mais auprès d'un autre groupe. Ils sont aussi solitaires.

\subsection*{Tests effectués}

Inspirés des caractéristiques des pigeons (qui peuvent être un type de boid particulier), nous avons commencés par paramétrer un a un et indépendamment les paramètres de ces oiseaux.

Ensuite nous avons recherché une harmonie entre ces différents paramètre lors du couplage, afin de reproduire visuellement des boids les plus proches de la réalité.

\end{document}
